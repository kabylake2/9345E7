

\documentclass[UTF8]{ctexart}
\usepackage[T1]{fontenc}
\usepackage{amsmath}
\usepackage{amssymb}
\usepackage{mathdots}
\setlength{\parindent}{0em}

\begin{document}

\section{常识}

正交阵 $A^{T}A=AA^{T}=E$, $A^{-1}=A^{T} , |A|^2 =1$


实对称 $A^{T}=A$

反对称 $A^{T}=-A$

A为实矩阵,若$A^{T}A=0\text{则}$$a_{i}=0,A=0$
$a_ij=A_ij \rightarrow A^*=A^T$ \\

正交向量内积为0

n阶行列式全部展开式总共n!项,每行取自不同行不同列的乘积

n$\geq$2时,n!是偶数

逆对角线正负号$(-1)^{\frac{n(n-1)}{2}}$

$(A^{*})^{*}=|A|^{n-2}A$


$|2En|=2^n$


$A+A^{T}$是对称矩阵,$A-A^{T}$是反对称矩阵

已知$a_{ij}=A_{ij} , \mbox{所以} A^* =A^T ,$

特征量不能为零

$r(A^TA)=r(A)$

可逆矩阵首先是方阵


\section{合同}

定义 C可逆,$B=C^{T}AC$

标准型 只有平方项

规范型 只有 0 1 -1

\section{范德蒙行列式}

$\begin{vmatrix}1 & 1 & \cdots & 1\\
x_{1} & x_{2} & \cdots & x_{3}\\
\vdots & \vdots & \vdots & \vdots\\
x_{1}^{n-1} & x_{2}^{n-2} & \vdots & x_{3}^{n-1}
\end{vmatrix}=\prod_{1\leq i<j\leq n}(x_{j}-s_{i})$

\section{拉普拉斯展开式}

$\begin{vmatrix}O & A_{n\times n}\\
B_{m\times m} & O
\end{vmatrix}$=$\begin{vmatrix}C & A\\
B & O
\end{vmatrix}=\begin{vmatrix}O & A\\
B & C
\end{vmatrix}$=$(-1)^{mn}|A||B|$

$\begin{array}{|cc|}
A_{mm} & O\\
O & B_{nn}
\end{array}$=$\begin{vmatrix}A & C\\
O & B
\end{vmatrix}=\begin{vmatrix}A & O\\
C & B
\end{vmatrix}=|A||B|$

A=$\begin{bmatrix} &  &  & A_{1}\\
 &  & A_{2}\\
 & \iddots\\
A_{n}
\end{bmatrix}$,$A^{-1}=\begin{bmatrix} &  &  & A_{n}^{-1}\\
 &  & \iddots\\
 & A_{2}^{-1}\\
A_{1}^{-1}
\end{bmatrix}$

$\begin{bmatrix}A & O\\
O & B
\end{bmatrix}^{-1}=\begin{bmatrix}A^{-1} & O\\
O & B^{-1}
\end{bmatrix}$

$\begin{bmatrix}A & O\\
O & B
\end{bmatrix}^{n}=\begin{bmatrix}A^{n} & O\\
O & B^{n}
\end{bmatrix}$

\section{初等变换}

单位矩阵经过一次初等变换称初等矩阵

初等矩阵都是可逆矩阵

可逆矩阵可表示一系列初等矩阵的乘积

\section{向量组等价}

等价向量组,相互线性表出

向量组和它的极大线性无关组是等价向量组 \\
两向量组等秩$\nrightarrow $两向量组等价  \\
向量组$a_{1},a_{2},a_{3},\ldots,a_{s}$线性无关,$a_{1},a_{2},a_{3},\ldots,a_{s},b$相关,b可由a1,a2,a3…as线性表出,表法唯一

向量组线性无关,增加一维的廷伸向量组无关

廷伸向量组相关,原向量组相关

含有零向量或成比例的向量组线性相关

\section{矩阵等价}

存在可逆矩阵PQ使$PAQ=B$,则$A\cong B$

两个同型矩阵等价的充分必要条件是秩相等

\section{相似}

\section{施密特正交化}

$\beta_{1}=\alpha_{1}$

$\beta_{2}=\alpha_{2}-\frac{(\alpha_{2},\beta_{1})}{(\beta_{1},\beta_{1})}\beta_{1}$

$\beta_{n}=\alpha_{n}-\frac{(\alpha_{n},\beta_{n-1})}{(\beta_{n-1},\beta_{n-1})}\beta_{n-1}-\frac{(\alpha_{n},\beta_{n-2})}{(\beta_{n-2},\beta_{n-2})}\beta_{n-2}-\cdots-\frac{(\alpha_{n},\beta_{1})}{(\beta_{1},\beta_{1})}\beta_{1}$

\section{迹}

迹Tr 方阵对角线所有元素之和

Tr(AB)=Tr(BA) , Tr(ABC)≠Tr(BAC)\_增加C不成立

\section{秩}

设A为m{*}n矩阵

r(AB)$\leq$min\{r(A),r(B)\}

AB=O r(A)+r(B)$\leq$n

$r(A)+r(B)\leq r\begin{pmatrix}A & O\\
C & B
\end{pmatrix}\leq r(A)+r(B)+r(C)$

r(A+B)<=r(A|B)<=r(A)+r(B)

r(A)<n-1,则|A|的全部代数余子式都为0

$|A*|=|A|^{n-1}$

r(A)=r 个 独立未知量 , n-r 个 自由未知量,解向量
\section{???}
$XTAX=0 \rightarrow A^T=-A$


\end{document}
