
% code=utf-8
\documentclass[UTF8]{ctexart}
\usepackage{amsmath}
\usepackage{graphicx}
\begin{document}
\section{点到直线}
$ {\left |{AX_0+BX_0+C} \right |} \over \sqrt{A^2+B^2} $

\section{和差化积}
$$ \sin \alpha + \sin \beta = 2 \sin { \left ( {\alpha + \beta } \over 2 \right ) } \cos \left ( {\alpha - \beta } \over 2 \right ) $$
$$\sin \alpha - \sin \beta = 2 \cos { \left ( {\alpha + \beta } \over 2 \right ) } \sin \left ( {\alpha - \beta } \over 2 \right ) $$
$$ \cos \alpha + \cos \beta = 2 \cos { \left ( {\alpha + \beta } \over 2 \right ) } \cos \left ( {\alpha - \beta } \over 2 \right )$$
$$ \cos \alpha - \cos \beta = -2 \sin { \left ( {\alpha + \beta } \over 2 \right ) } \sin \left ( {\alpha - \beta } \over 2 \right )$$

\section{积化和差}
$$ \sin \alpha \cos \beta = \frac{1}{2} \left [ \sin(\alpha+\beta)+\sin(\alpha-\beta) \right ]$$
$$ \cos \alpha \sin \beta = \frac{1}{2} \left [ \sin(\alpha+\beta)-\sin(\alpha-\beta) \right ]$$
$$ \cos \alpha \cos \beta = \frac{1}{2} \left [ \cos(\alpha+\beta)+\cos(\alpha-\beta) \right ]$$
$$ \sin \alpha \sin \beta =- \frac{1}{2} \left [ \cos(\alpha+\beta)-\cos(\alpha-\beta) \right ]$$

\section{万能公式}
$$ \sin\alpha=\frac{2\tan \frac{\alpha}{2}}{1+\tan^2 \frac{\alpha}{2}}$$
$$ \cos \alpha =\frac {1-\tan^2\frac{\alpha}{2}}{1+\tan^2\frac{\alpha}{2}}$$
$$ \tan\alpha=\frac{2\tan \frac{\alpha}{2}}{1-\tan^2 \frac{\alpha}{2}}$$

\section{半角公式}
$$ \sin \left(\frac{\alpha}{2} \right)=\pm \sqrt{\left(\frac{1-cos \alpha}{2}\right)}$$
$$ \cos \left(\frac{\alpha}{2} \right)=\pm \sqrt{\left(\frac{1+cos \alpha}{2}\right)}$$
$$ \tan \left(\frac{\alpha}{2} \right)=\pm \sqrt{\left(\frac{1-cos \alpha}{1+cos \alpha}\right)}$$

\section{函数对称}
f(×)关于x=T对称  充要条件 \\
f(x)=f(2T-×)  ;f(T+x)=f(T-x)

\section{奇函数与偶函数的表达}
奇 F(x)=f(x)-f(-x) \\
偶 F(x)=f(x)+f(-x) \\
任意 f(x)=1/2[f(x)-f(-x)]+1/2 f(x)+f(-x)]

\section{最大值\&最小值}
Max{f(x),g(x)}=1/2[f(x)+g(x) +|f(x)-g(x)|] \\
Min{f(x),g(x)}=1/2[f(x)+g(x)-|f(x)-g(x)|]

\section{反函数}
f(x)  g(x)互为反函数 \\
f(g(x))=x $\rightarrow$  g(f(x))

\section{数列敛散性}
数列收敛于A,则任意子数列收剑于A   \\
单调数列的某一子数列收敛于A,则该数列收敛于A \\
数列{2n}与{2n+1}都收敛于A,则数列必收敛于A

\section{连续的定义}
$$ \lim_{x\rightarrow a}f(x)=A$$
$$ \lim_{\Delta\rightarrow0}\Delta y=\lim_{\Delta x \rightarrow 0 } f(x+ \Delta x) - f(x)=0$$

\section{常用等价无穷小}
 $ x \rightarrow 0$
$ \sin x \sim x$ ; $ \tan x \sim x$ ; $ \arcsin x \sim x$ ; $ \arctan x \sim x$ ; $ \ln({1+x}) \sim x $ ; $ e^x -1 \sim x$ ; $ a^x -1 \sim x \ln a $ ; $ 1-\cos x \sim \frac{1}{2} x^2 $ ; $ {(1+x)}^a -1 \sim ax$ \\

f(0)=1 时 等价无穷小 \\
$$ \lim_{x \rightarrow 0 } {{\int_0^x f(x)dt} \over {x}} =1$$

\section{极限比较}
$$ f(x) \geq g(x) \rightarrow \lim f(x) \geq \lim g(x) $$
$$ \lim f(x) > \lim g(x) \rightarrow f(x) > g(x) $$
\section{敛散}
$ \lim_{x \rightarrow 1} \frac{1}{(x-1)^{α+1}}=
  \begin{cases}
  &0 , (a<-1) \\
  &1 , (a=-1) \\
  &\infty ,(a>-1)
\end{cases}
$
$$ \frac{\partial f}{\partial x} \equiv \frac{\partial f}{\partial y} \equiv 0\ \leftrightarrow df({x,y}) \equiv 0 $$

\section{基本函数求导公式}
$ (a^x)'=a^x \ln a \\
  ( \log _a x)'=\frac{1}{x \ln a} $

\section{球体积\&表面积}
$$ V=\frac{4}{3} \pi R^3$$
$$ S= 4 \pi R^2$$

\section{定积分定义}
$$ \int_a^b f(x)dx= \lim_{n \rightarrow \infty } \sum_{i=1}^n \frac{f \left( a+ \frac{b-a}{n}i \right)(b-a)}{n}$$
\section{连续函数必有原函数}
含有第一类间断点,无穷间断点的函数在包含间断点的区间没有原函数   \\
跳跃间断点可以有原函数 \\

\section{基本积分表}
\includegraphics[width=14cm]{9345E7/9476215E0863B8FB681EC9D2BB921BDB.jpg}
\includegraphics[width=14cm]{9345E7/4FEE9C843E57935F87D1E16F255FA43E.jpg}

$$ \int \tan x dx = - \ln |\cos x | +C$$
$$ \int cot x dx = \ln|\sin x| +C$$
$$ \int \sec x dx = \ln |\sec x+ \tan x| +C$$
$$ \int \csc x dx = \ln|csc x-\cot x | +C $$
$$ \int \frac{dx}{a^2+x^2}=\frac{1}{a} \arctan \frac{x}{a} +C $$
$$ \int \frac{dx}{x^2-a^2}=\frac{1}{2a} \ln \left| \frac{x-a}{x+a} \right| +C$$
$$ \int \frac{dx}{\sqrt{a^2-x^2}}=\arcsin \frac{x}{a}+C$$
$$ \int \frac {dx}{\sqrt{x^2+a^2}}=\ln {(x+ \sqrt{x^2+a^2})}+C$$
$$ \int \frac{dx}{\sqrt{x^2-a^2}}= \ln | x+\sqrt{x^2-a^2}|+C$$

\section{泰勒公式}
$ e^{1+x}=e+ex+e \frac{x^2}{2!} + e \frac{x^3}{3!}+e\frac{x^4}{4!}$

\includegraphics[width=14cm]{9345E7/F04F85A8EB6EABFF18D6BE71383F2472.jpg}
$$ \sin x=x-\frac{x^3}{3!}+o(x^3)$$
$$ \cos x=1-\frac{x^2}{2!}+\frac{x^4}{4!}+o(x^4)$$
$$ \arcsin x=x+\frac{x^3}{3!}+o(x^3)$$
$$ \tan x=x+\frac{x^3}{3}+o(x^3)$$
$$ \arctan x=x-\frac{x^3}{3}+o(x^3)$$

\section{几个初等函数的n阶导数公式}
\includegraphics[width=13cm]{9345E7/2A793F093B002F668B144F9ED087EB77.jpg}


$$ \int_0^\frac{\pi}{2} sin^n θ d \theta =\int_0^\frac{\pi}{2} \cos^n \theta d \theta =
\begin{cases}
  &\frac{n-1}{n}\cdot\frac{n-3}{n-2} \cdots \frac{3}{4}\cdot\frac{1}{2}\cdot\frac{\pi}{2}\mbox{  n为正偶数} \\
  &\frac{n-1}{n}\cdot\frac{n-3}{n-2}\cdots\frac{4}{5}\cdot\frac{2}{3}\mbox{  n为大于1的正奇数}
\end{cases}
$$

\section{经典不等式}
$$e^x \geq x+1 ; x-1 \geq \ln x ; \frac{1}{1+x} < \ln \left( 1+ \frac{1}{x} \right) < \frac{1}{x}$$
$$ e^{αx} \gg x^b \gg \ln^y x $$
$$ 2\left| ab \right| \leq a^2+b^2$$
$$ \left| a \pm b \right| \leq |a|+|b|$$
$$ \left| |a| - |b| \right| \leq |a-b|$$
$$\sqrt{ab} \leq \frac{a+b}{2} \leq \sqrt{\frac{a^2+b^2}{2}}$$
当 x>0,y>0,p>0,q>0,$\frac{1}{p}+\frac{1}{q}=1 \rightarrow xy \leq \frac{x^p}{p}+\frac{y^q}{q}$
$$(a^2+b^2)(c^2+d^2) \geq (ac+bd)^2$$
$$ [\int_a^b f(x)g(x)dx]^2 \leq \int_a^b f^2(x)dx\cdot \int_a^b g^2(x)dx$$
$$\mbox{当} p>1, \frac{1}{p}+\frac{1}{q}=1 \mbox{时;} \left| \int_a^b f(x) \cdot g(x)dx\right| \leq \left[ \int_a^b \left| f(x) \right|^p dx \right] ^\frac{1}{p} \cdot \left[ \int_a^b \left| g(x) \right|^q dx \right] ^\frac{1}{q}$$

​
1. 若$\int^{(n−1)}(x)$最多只有一个实零点,则f(x)最多只有n个不同实零点
1. $f^′(x)≠0$ 且连续⇒ f(x)单调
1. 连续的**奇**函数的**一  切**原函数都是**偶**函数
1. 连续的**偶**函数的**仅有一个**原函数都是**奇**函数
1. 可积函数在区间内必有界 (二元也成立)
1. f(x)是以ㄒ为周期的可积函数


\end{document}
